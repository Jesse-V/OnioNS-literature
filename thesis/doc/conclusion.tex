
\chapter{CONCLUSION}

\vspace{24pt}

We have introduced OnioNS, a Tor-powered distributed DNS that maps unique .tor domain names to traditional Tor .onion addresses. It enables hidden service operators to select a human-meaningful domain name and provide access to their service through that domain. We preserve the privacy and anonymity of both parties during registration, maintenance, and lookup, and furthermore allow Tor clients to verify the authenticity of domain names. Moreover, we rely heavily upon existing Tor infrastructure, which simplifies our design assumptions and narrows our threat model largely to attack vectors already well-understood throughout the Tor literature.

We use the Pagechain distributed data structure to prevent disagreements from forming within the network. Furthermore, every participant can verify the uniqueness of domain names. The Pagechain also has a fixed maximal length, which places an upper bound on the networking, computational, and storage requirements for all participants, a valuable efficiency gain especially noticeable long-term.

OnioNS achieves all three properties of Zooko's Triangle: it is distributed, allows hidden service operators to select meaningful domain names, and all parties can confirm for themselves the uniqueness of domain names in the database. We provide a reference implementation in C++ that should enable Tor developers to deploy OnioNS into the Tor network with minimal effort. We believe that OnioNS will be a useful abstraction layer that will significantly enhance the usability and the popularity of Tor hidden services. 