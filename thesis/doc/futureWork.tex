
\chapter{FUTURE WORK}

In future work we will expand our implementation of OnioNS and develop the remaining protocols. While our implementation functions with a fixed resolver, we will deploy our implementation onto larger and more realistic simulation environments such as Chutney and PlanetLab. When we have completed dynamic functionality and the remaining protocols, we will pursuit integrating our implementation into Tor. We expect this to be straightforward as OnioNS is designed as a plugin for Tor, introduces no changes to Tor's hidden service protocol, and requires very few changes to Tor's software. In future developments, Tor's developers may also make significant changes to Tor's hidden services, but OnioNS' design enables our system to become forwards-compatible after a few minor changes.

Additionally, several questions need to answered in future studies:

\begin{itemize}
	\item Should the Quorum expire registrations that point to non-existent hidden services, and if so, how can this be done securely?
	\item How can we reduce vulnerability to phishing/spoofing attacks? Can OnioNS be adapted to include a privacy-enhanced reputation system?
	\item How can OnioNS support domain names with international encodings? A na\"{i}ve approach to this is to simply support UTF-16, though care must also be taken to prevent phishing attacks by domain names that use Unicode characters that visually appear very similar.
	\item What other networks can OnioNS apply to? We require a fully-connected networked and an global source of entropy. We encourage the community to adapt our work to other systems that fit these requirements.
\end{itemize}