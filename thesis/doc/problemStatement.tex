
\chapter{PROBLEM STATEMENT}

\section{Assumptions and Threat Model}
\label{sec:Assumptions}

OnioNS' basic design and protocols rely on several assumptions and expected threat vectors.

\begin{enumerate}
	\item We assume that Tor provides privacy and anonymity; if Alice constructs a three-hop Tor circuit to Bob with modern Tor circuit construction protocols and sends a message $ m $ to Bob, we assume that Bob can learn no more about Alice than the contents of $ m $. This implies that if $ m $ does not contain identifiable information, Alice is anonymous from Bob's perspective, regardless of if $ m $ is exposed to an attacker, Eve. Identifiable information in $ m $ is outside of Tor's scope, but we do not introduce any protocols that cause this scenario.

	\item We assume secure cryptographic primitives; namely that Eve cannot break standard cryptographic primitives such as AES, SHA-2, RSA, Curve25519, Ed25519, and the scrypt key derivation function. We assume that Eve maintains no backdoors or knows secret software breaks in the Botan or the OpenSSL implementations of these primitives.

	\item We assume that not all Tor routers are honest; that Eve controls some percentage of Tor routers such that Eve's routers may actively collude. Routers may also be semi-honest; wiretapped but not capable of violating protocols. However, the percentage of dishonest and semi-honest routers is small enough to avoid violating our first assumption. We assume a fixed percentage of dishonest and semi-honest routers; namely that the percentage of routers under an Eve's control does not increase in response to the inclusion of OnionNS into Tor infrastructure. This assumption simplifies our threat model analysis but we consider it realistic because while Tor traffic is purposely secret as it travels through the network, we consider OnioNS information public so we don't consider the inclusion of OnioNS a motivating factor to Eve.

	\item If $ C $ is a Tor network status consensus, $ Q $ is an $ M $-sized set randomly but deterministically selected from the Fast and Stable routers listed in $ C $, and $ Q $ is under the influence of one or more adversaries, we assume that the largest subset of agreeing routers in $ Q $ are at least semi-honest.
		
\end{enumerate}

\section{Design Objectives}
\label{sec:Objectives}

Here we enumerate a list of requirements that must be met by any DNS applicable to Tor hidden services. In Chapter \ref{ch:ExistingWorks} we analyse several existing prominent naming systems and show how these systems do not meet these requirements. In Chapters \ref{ch:Solution} and \ref{ch:Analysis} we demonstrate how we overcome them with OnioNS.

\begin{enumerate}
	\item \textbf{The system must support anonymous registrations.} The system should not require any personally-identifiable or location information from the registrant.
	\item \textbf{The system must support privacy-enhanced queries.} Clients should be anonymous, indistinguishable, and unable to be tracked by name servers.
	\item \textbf{Registrations must be authenticable.} Clients must be able to verify that the domain-address pairing that they receive from name servers is authentic relative to the authenticity of the hidden service. % todo: OnionNS should show the resolved .onion to confirm this
	\item \textbf{Domain names must be globally unique.} Any domain name of global scope must point to at most one server. For naming systems that generate names via cryptographic hashes, the key-space must be of sufficient length to resist cryptanalytic attack.
	\item \textbf{The system must be distributed.} Systems with root authorities have distinct disadvantages compared to distributed networks: specifically, central authorities have absolute control over the system and root security breaches could easily compromise the integrity of the entire system. Root authorities may also be able to compromise the privacy of both users and hidden services or may not allow anonymous registrations.
	\item \textbf{The system must be relatively easy to use.} It should be assumed that users are not security experts or have technical backgrounds. The system must resolve protocols with minimal input from the user and hide non-essential details.
	\item \textbf{The system must be backwards compatible.} Naming systems for Tor must preserve the original Tor hidden service protocol, making the DNS optional but not required.
	\item \textbf{The system should be lightweight.} In most realistic environments clients have neither the bandwidth nor storage capacity to hold the system's entire database, nor the capability of meeting significant computation burdens.
\end{enumerate}

