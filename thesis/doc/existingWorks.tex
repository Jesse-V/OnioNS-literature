
\chapter{EXISTING WORKS}
\label{ch:ExistingWorks}

\vspace{12pt}

We now examine several workarounds and existing naming systems against the design objectives listed in Section \ref{sec:Objectives}.

\section{Address Manipulation}

Although they are not separate naming systems in their own right, several systems have been proposed to directly improve the readability of hidden service addresses. These include vanity key generators and different encoding schemes.

Shallot is a vanity key generator which generates by brute-force many RSA keys in attempt to find one that has a desirable hash\cite{KatmagicShallot}. For example, a hidden service operator may wish to start his service's address with a meaningful noun so that others may more easily recognize it. Shallot has been used successfully for both common and high-profile hidden services, such as blockchainbdgpzk.onion, an official mirror of Blockchain.info; facebookcorewwwi.onion, Facebook's hidden service; and  freepress3xxs3hk.onion, the Freedom of the Press's SecureDrop instance. However, Shallot is only partially successful at enhancing readability because the size of the domain key-space is too large to be fully brute-forced in any reasonable length of time\cite{KatmagicShallot}. This situation is expected to get worse over time as Tor plans to increase the length of hidden service addresses\cite{Proposal224}. If the address key-space was reduced to allow a full brute-force, the system would fail to be collision-free.

Nicolussi suggested changing the address encoding from base32 to a delimited series of words, using a dictionary known in advance by all parties\cite{nicolussi2011human}. Like Shallot, Nicolussi's encoding is cosmetic and only partially improves the recognition and readability of an address but does nothing to alleviate the logistic problems of manually entering in the address into the Tor Browser.

\section{Centralized or Zone-Based DNS}

\subsection{Internet DNS}

The Internet DNS is another one candidate and is already well established as a fundamental abstraction layer for Internet routing. Despite its widespread use and extreme popularity, the Internet DNS suffers from several significant shortcomings and security issues that make it inappropriate for use by Tor hidden services. With the exception of extensions such as DNSSEC, the Internet DNS by default does not use any cryptographic primitives. DNSSEC is primarily designed to prevent forgeries and DNS cache poisoning from intermediary name servers and it does not provide any degree of query privacy\cite{wachs2014censorship}. Additional extensions and protocols such as DNSCurve\cite{bernstein2009dnscurve} have been proposed, but DNSSEC and DNSCurve are optional and have not yet seen widespread full deployment across the Internet. Traditional DNS lookups may be intercepted and modified by MITM attacks, user privacy may be compromised by wiretapping DNS lookups, and the system is by default vulnerable to DNS cache poisoning.

The lack of default security in Internet DNS and the financial expenses involved with registering a new TLD casts significant doubt on the feasibility of using it for Tor hidden services. Furthermore the system meets only a few of our design requirements: although the system is easy to use, the system is hierarchical but not truly distributed, domain registrars typically require the owner to reveal significant amounts of identifiable information, registrations are not confirmable except through expensive SSL certificates issues by a central authority, and lookups occur by default without any privacy enhancements. These issues make the Internet DNS ill-suited for Tor hidden services.

\subsection{GNU Name System}

The GNU Name System\cite{wachs2014censorship} (GNS) is a zone-based alternative DNS. GNS describes a hierarchical zones of names (using the .gns pseudo-TLD) with each user managing their own zone and distributing zone access peer-to-peer within social circles. While GNS' design guarantees the uniqueness of names within each zone and users are capable of selecting meaningful nicknames for themselves, GNU does not guarantee that names are \emph{globally} unique. Furthermore, the selection of a trustworthy zone to use would be a significant challenge for using GNS for Tor hidden services and such a selection no longer makes the system distributed. However, GNS does meet many, but not all, of our requirements, so we consider GNS a very impressive system and recommend GNS as a possible fallback from OnioNS.

\subsection{Namecoin}
\label{sec:Namecoin}

Namecoin is an early fork of Bitcoin\cite{nakamoto2008bitcoin} and is noteworthy for achieving all three properties of Zooko's Triangle. Namecoin holds digitally-signed information transactions in a data structure known as a block; each block links to a previous block, forming a public ledger known as a blockchain. Storing textual information such as a domain registration consumes some Namecoins, a unit of currency. In 2014, Namecoin was recognized by ICANN as the most well-known example of a PKI and DNS system with an emphasis of distributed control and privacy.

While Namecoin is often advertised as capable of assigning names to Tor hidden services, it has several practical issues that make it generally infeasible to be used for that purpose. First, to authenticate registrations, clients must be able to prove the relationship between a Namecoin owner's secp256k1 ECDSA key and the target hidden service's RSA key, and constructing this relationship is non-trivial. Second, Namecoin is typically a heavyweight DNS: it generally requires users to pre-fetch and then verify the blockchain, which is 2.45 GB as of April 2015\cite{BitInfoCharts}. Third, although Namecoin supports anonymous ownership of information, it is non-trivial to anonymously purchase Namecoins, thus preventing domain registration from being truly anonymous. These issues prevent Namecoin from being a practical alternative DNS for Tor hidden service. However, our work shares some design principles with Namecoin.
