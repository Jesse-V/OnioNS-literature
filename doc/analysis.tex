
\chapter{ANALYSIS}

%and the visibility of the final

We have designed OnionNS to meet our original design goals. Assuming that Tor circuits are a sufficient method of masking one's identity and location, hidden service operators can perform operations on their records anonymously. Likewise, a Tor circuit is also used when a client lookups a .tor domain name, just as Tor-protected DNS lookups are performed when browsing the Clearnet through the Tor Browser. OnionNS records are self-signed and include the hidden service's public key, so anyone --- particularly the client --- can confirm the authenticity (relative to the authenticity of the public key) and integrity of any record. This does not entirely prevent Sybil attacks, but this is a very hard problem to address in a distributed environment without the confirmation from a central authority. However, the proof-of-work component makes record spoofing a costly endeavour, but it is not impossible to a well-resourced attacker with sufficient access to high-end general-purpose hardware.

Without complete access to a local copy of the database a party cannot know whether a second-level domain is in fact unique, but by using an existing level of trust with a known network they can be reasonably sure that it meets the unique edge of Zooko's Triangle. Anyone holding a copy of the consensus document can generate the set of \emph{quorum} node and verify their signatures. As the \emph{quorum} is a set of nodes that work together and the \emph{quorum} is chosen randomly from reliable nodes in the Tor network, OnionNS is a distributed system. Tor clients also have the ability to perform a full synchronization and confirming uniqueness for themselves, thus verifying that Zooko's Triangle is complete. Hidden service .onion addresses will continue to have an extremely high chance of being securely unique as long the key-space is sufficiently large to avoid hash collisions.

Just as traditional Clearnet DNS lookups occur behind-the-scenes, OnionNS Domain Queries require no user assistance. Client-side software should filter TLDs to determine which DNS system to use. We introduce no changes to Tor's hidden service protocol and also note that the existence of a DNS system introduces forward-compatibility: developers can replace hash functions and PKI in the hidden service protocol without disrupting its users, so long as records are transferred and OnionNS is updated to support the new public keys. We therefore believe that we have met all of our original design requirements.

\section{Security}

\subsection{Quorum-level Attacks}

The quorum nodes hold the greatest amount of responsibility and control over OnionNS out of all participating nodes in the Tor network, therefore ensuring their security and limiting their attack capabilities is of primary importance.

\subsubsection{Passively Malicious Quorum}

In section \ref{sec:Assumptions}, we assumed that an attacker, Eve, already controls a fraction of Tor routers. For a dynamic network size, a fixed fraction of dishonest nodes, and a fixed quorum size, every time a quorum is selected there is a probability that more than half of the quorum is dishonest and is colluding together. If this occurs, records may be dropped rather than archived. \emph{Here we explore the optimal quorum rotation rate, given the balance between high overhead and high security risk if the quorum is rotated quickly, and long-term implications and possible stability issues associated with very slow rotation.}

\subsubsection{Active Malicious Quorum}

If Eve controls some Tor nodes (who may be assumed to be colluding with one another), the attacker may desire to include their nodes in the quorum for malicious manipulation, passive observation, or for other purposes. Alternatively, Eve may wish to exclude certain legitimate nodes from inclusion in the quorum. In order to carry out either of these attacks, Eve must have the list of qualified Tor nodes scrambled in such a way that the output is pleasing to Eve. Specifically, the scrambled list must contain at least some of Eve's malicious nodes for the first attack, or exclude the legitimate target nodes for the second attack. We initialize Mersenne Twister with a 384-bit seed, thus Eve can find $ k $ seeds that generates a desirable scrambled list in $ 2^{192} $ operations on average, or $ 2^{384} $ operations in the worst case. The chance of any of those seeds being selected, and thus Eve successfully carrying out the attack, is thus $ \frac{2^{384}}{k} $.

Eve may attempt to manipulate the consensus document in such a way that the SHA-384 hash is one of these $ k $ seeds. Eve may instruct her Tor nodes to upload a custom status report to the authority nodes in an attempt to maliciously manipulate the contents of the consensus document, but SHA-384's strong preimage resistance and the unknown state and number of Tor nodes outside Eve's control makes this attack infeasible. The time to break preimage resistance of full SHA-384 is still $ 2^{384} $ operations. This also implies that Eve cannot determine in advance the next consensus document, so the new quorum cannot be predicted. If Eve has compromised at least some of the Tor authority nodes she has significantly more power in manipulating the consensus document for her own purposes, but this attack vector can also break the Tor network as a whole and is thus outside the scope of our analysis. Therefore, the computation required to maliciously generate the quorum puts this attack vector outside the reach of computationally-bound adversaries.

OnionNS and the Tor network as a whole are both susceptible to Sybil attacks, though these attacks are made significantly more challenging by the slow building of trust in the Tor network. Eve may attempt to introduce large numbers of nodes under her control in an attempt to increase her chances of at least one of the becoming members of the \emph{quorum}. Sybil attacks are not unknown to Tor; in December 2014 the black hat hacking group LizardSquad launched ~3000 nodes in the Google Cloud in an attempt to intercept the majority of Tor traffic. However, as Tor authority nodes grant consensus weight to new Tor nodes very slowly, despite controlling a third of all Tor nodes, these 3,000 nodes moved 0.2743 percent of Tor traffic before they were banned from the Tor network. The Stable and Fast flags are also granted after weeks of uptime and a history of reliability. As nodes must have these flags to be qualified as a \emph{quorum} \emph{candidate}, these large-scale Sybil attacks are financially demanding and time-consuming for Eve.

\subsection{Non-existence Forgery}

As we have stated earlier, falsely claiming a negative on the existence of a record is a problem overlooked in other domain name systems. One of the primary challenges with this approach is that the space of possible names so vast that attempting to enumerate and digitally sign all names that are not taken is highly impractical. Without a solution, this weakness can degenerate into a denial-of-service attack if the DNS resolver is malicious towards the client. Our counter-measure is the highly compact hashtable bitset with a Merkle tree for collisions. We set the size of the hashtable such that the number of collisions is statistically very small, allowing an efficient lookup in $ \mathcal{O}(1) $ time on average with minimal data transferred to the client.

\subsection{Name Squatting and Record Flooding}

An attacker, Eve, may attempt a denial-of-service attack by obtaining a set of names for the sole purpose of denying them to others. Eve may also wish to create many name requests and flood the \emph{quorum} with a large quantity of records. Both of these attacks are made computationally difficult and time-consuming for Eve because of the proof-of-work. If Eve has access to large computational resources or to custom hardware she may be able to process the PoW more efficiently than legitimate users, and this can be a concern.

The proof-of-work scheme is carefully designed to limit Eve to the same capabilities as legitimate users, thus significantly deterring this attack. The use of scrypt makes custom hardware and massively-parallel computation expensive, and the digital signature in every record forces the hidden service operator to resign the fields for every iteration in the proof-of-work. While the scheme would not entirely prevent the operator from outsourcing the computation to a cloud service or to a secondary offline resource, the other machine would need the hidden service private key to regenerate \emph{recordSig}, which the operator can't reveal without compromising his security. However, the secondary resource could perform the scrypt computations in batch without generating \emph{recordSig}, but it would always perform more than the necessary amount of computation because it would could not generate the SHA-384 hash and thus know when to stop. Furthermore, offloading the computation would still incur a cost to the hidden service operator, who would have to pay another party for the consumed computational resources. Thus the scheme always requires some cost when claiming a domain name.

\section{Performance}

bandwidth, CPU, RAM, latency for clients to be determined...

\subsection{Load}

demand on participating nodes to be determined...

Unlike Namecoin, OnionNS' \emph{page}-chain is of $ L $ days in maximal length. This serves two purposes:

\begin{enumerate}
	\item Causes domain names to expire, which reduced the threat of name squatting.
	\item Prevents the data structure from growing to an unmanageable size.
\end{enumerate}

\section{Reliability}

%  on Unreliable Hosts

Tor nodes have no reliability guarantee and may disappear from the network momentarily or permanently at any time. Old \emph{quorums} may disappear from the network without consequence of data loss, as their data is cloned by current \emph{mirrors}. So long as the \emph{quorum} nodes remain up for the $ \Delta i $ days that they are active, the system will suffer no loss of functionality. Nodes that become temporarily unavailable will have out-of-sync \emph{pages} and will have to fetch recent records from other \emph{quorum} nodes in the time of their absence.



