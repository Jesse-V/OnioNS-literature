

\chapter{\uppercase{Analysis}}

general analysis...

\section{Security}

\subsection{Quorum-level Attacks}

The quorum nodes hold the greatest amount of responsibility and control over EsgalDNS out of all participating nodes in the Tor network, therefore ensuring their security and limiting their attack capabilities is of primary importance.

\subsubsection{Malicious Quorum Generation}

If an attacker, Eve, controls some Tor nodes (who may be assumed to be colluding with one another), the attacker may desire to include their nodes in the quorum for malicious manipulation, passive observation, or for other purposes. Alternatively, Eve may wish to exclude certain legitimate nodes from inclusion in the quorum. In order to carry out either of these attacks, Eve must have the list of qualified Tor nodes scrambled in such a way that the output is pleasing to Eve. Specifically, the scrambled list must contain at least some of Eve's malicious nodes for the first attack, or exclude the legitimate target nodes for the second attack. We initialize Mersenne Twister with a 384-bit seed, thus Eve can find $ k $ seeds that generates a desirable scrambled list in $ 2^{192} $ operations on average, or $ 2^{384} $ operations in the worst case. The chance of any of those seeds being selected, and thus Eve successfully carrying out the attack, is thus $ \frac{2^{384}}{k} $.

Eve may attempt to manipulate the consensus document in such a way that the SHA-384 hash is one of these $ k $ seeds. Eve may instruct her Tor nodes to upload a custom status report to the authority nodes in an attempt to maliciously manipulate the contents of the consensus document, but SHA-384's strong preimage resistance and the unknown state and number of Tor nodes outside Eve's control makes this attack infeasible. As of the time of this writing, the best preimage break of SHA-512 is only partial (57 out of 80 rounds in $ 2^{511} $ time\cite{li2012converting}) so the time to break preimage resistance of full SHA-384 is still $ 2^{384} $ operations. This also implies that Eve cannot determine in advance the next consensus document, so the new quorum cannot be predicted. If Eve has compromised at least some of the Tor authority nodes she has significantly more power in manipulating the consensus document for her own purposes, but this attack vector can also break the Tor network as a whole and is thus outside the scope of our analysis. Therefore, the computation required to maliciously generate the quorum puts this attack vector outside the reach of computationally-bound adversaries.

Requirements for stable and fast nodes makes spinning up lots of nodes (LizardSquad) more challenging.

\section{Performance}

bandwidth, CPU, RAM, latency for clients..

\subsection{Load}

demand on participating nodes...

