
\chapter{PROBLEM STATEMENT}

\section{Assumptions and Threat Model}
\label{sec:Assumptions}

OnioNS' basic design and protocols rely on several assumptions and expected threat vectors.

\begin{enumerate}
	\item Let Alice is a Tor user and let Bob be a recipient. If Alice constructs a three-router Tor circuit via NTor and sends a message $ m $ to Bob, we assume that the Tor circuit preserves Alice's privacy and provides her with anonymity from Bob's perspective. Namely, we assume that out of Alice's identity, location, and knowledge of Bob, an outside attacker Eve can obtain at most one, and that Bob can know no more about Alice than the contents of $ m $. The exception to this is if Alice chooses to disclose her identity only to Bob, but then Bob does not know Alice's location. We assume that neither Bob nor Eve can force Alice into breaking her privacy involuntarily. The protection of a Tor circuit relies on Tor's assumption that Eve only has access to some Tor traffic; no defence is made by Tor nor by OnioNS to defend against a global adversary.
	\item Adversaries cannot break standard cryptographic primitives.
		\begin{enumerate}
			\item We assume that they cannot efficiently break AES ciphers, SHA-384 hashes, RSA-1024, Curve25519 ECDHE, Ed25519 digital signatures, or have hardware-level attacks against the scrypt key derivation function.
			\item We assume that they maintain no backdoors or other software breaks in the Botan or the OpenSSL implementations of the primitives.
			\item We assume that they are not capable of breaking cryptographic protocols built on the primitives such as TAP and NTor.
		\end{enumerate}
	\item We assume that not all Tor nodes are honest. We assume that at least some Tor nodes are run by malicious operators, curious researchers, experimenting developers, or government organizations. They may also be wiretapped, exploited, broken into, or become unavailable. This assumption is also made by Tor's developers and therefore we must conclude that any new functionality added to existing Tor nodes must also fall under their assumption. However, we assume that the majority of Tor nodes are honest and reliable. We consider this reasonable because it is a prerequisite for the security of Tor circuits.
	\item We assume that the percentage of dishonest routers within the Tor network does not increase in response to the inclusion of OnionNS into Tor infrastructure. This assumption simplifies our threat model analysis but we consider it realistic because Tor traffic is purposely kept secret and is therefore much more valuable to Eve. OnioNS uses public data structures so we don't consider the inclusion of OnioNS a motivating factor to Eve. Moreover, we assume that Eve cannot predict in advance the identities of the Tor routers that act as authorities for OnioNS.
	\item Let $ C $ be the set of Tor nodes that have the Fast and Stable flags and let $ Q $ be an $ M $-sized set chosen randomly from $ C $. $ Q $ may be under the influence of one or more adversaries who may cause subsets of $ Q $ to collude, but our final assumption is that the largest subset of $ Q $ is acting honestly according to specifications.
\end{enumerate}

\section{Design Objectives}

Tor's high security environment introduces a distinct set of challenges that must be met by additional functionality. Privacy and anonymity are of paramount importance. Here we enumerate a list of requirements for any security-enhanced DNS applicable to Tor hidden services. In Chapter \label{Chapter:ExistingWorks} we analyse several existing prominent naming systems and show how these systems do not meet these requirements. In Chapters \label{Chapter:Solution} and \label{Chapter:Analysis} we demonstrate how we overcome them with OnioNS.

\begin{enumerate}
	\item \textbf{The system must support anonymous registrations.} It must be hard for any party to infer the identity or location of the registrant, unless the registrant chooses to disclose that information. Hidden services are privacy-enhanced servers and the registration of a domain name should not compromise that privacy.
	\item \textbf{The system must support privacy-enhanced lookups.} The resolver should not be able to identify a client nor track their lookups. In other words, clients should ideally have anonymity and be indistinguishable from other clients from the perspective of a resolver.
	\item \textbf{Clients must be able to authenticate registrations.} The users of the system must be able to verify that the information they received from the service has not been forged and is authenticate relative to the authenticity of the server they will connect to. % todo: OnionNS should show the resolved .onion to confirm this
	\item \textbf{Domain names must be provably or have a near-certain chance of being unique.} Any domain name of global scope must point to at most one server. In the case of naming systems that generate names via cryptographic hashes, the domain name key-space must be of sufficient length to be remain resistant to at least collision and second pre-image attacks.
	\item \textbf{The system must be distributed.} Systems with root authorities have distinct disadvantages compared to distributed networks: specifically, central authorities have absolute control over the system and root security breaches could easily compromise the integrity of the entire system. Root authorities may also be able to easily compromise user privacy or may not allow anonymous registrations, although this is system dependent. For these reasons, naming systems with central authorities can safely be considered dangerous and ill-suited for hidden services.
	\item \textbf{The system must be simple and relatively easy to use.} It should be assumed that users are not security experts or have technical backgrounds. Tor's developers go to great lengths to ensure that Tor tools are straightforward, and the success of Tor's low-latency onion routing model demonstrates the effectiveness of convenience within high-security settings. Any naming system used by Tor must be equivalently simple, resolve protocols with minimal input from the user, and hide non-essential details.
	\item \textbf{The system must be backwards compatible with existing protocols.} Just as the Internet's DNS did not introduce any changes in the TCP/IP layer, naming systems for Tor must preserve the original Tor hidden service protocol. DNS is optional, not required.
\end{enumerate}

Additionally, we specify several performance-enhancing objectives, although these are not requirements.

\begin{enumerate}
	\item \textbf{The system should not require clients to download the entire database.} It can safely assumed that in most realistic environments clients have neither the network capacity nor the storage to hold the system's database. While they may choose to download the database, it should not be required for normal functionality or to meet any of the above objectives.
	\item \textbf{The system should not introduce significant burdens to the clients.} Despite the rapid and exponential growth of consumer-grade hardware, in most environments not every client is on a high-end machine. Therefore the system should not burden user computers with significant computation or memory demands.
	\item \textbf{The system should have low latency} and resolve lookup queries within a reasonable amount of time.
\end{enumerate}
