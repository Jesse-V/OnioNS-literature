% Be sure to write chapter titles in ALL CAPS
\chapter{\uppercase{Conclusions}}

Combinatorial testing is a powerful tool in identifying software faults.  Indeed, Kuhn, et al. showed in a study of a NASA Distributed Database that 93\% of all faults were identified by 2-way combinations, and 98\% by 3-way combinations.  Across industries, ``the detection rate curves for the other applications studied are similar, reaching 100\% detection with 4 to 6-way interactions'' \cite{kuhn2010practical}.  

Rich web applications continue to grow with the introduction and rapid development of new HTML5 features and APIs, powerful JavaScript based frameworks, and increasingly more powerful client machines.  This thesis demonstrates that combinatorial testing can play an important role in the testing of rich web applications and that more future work is needed.

While there is not a specific standard for semantic URLs compared to the traditional URL format defined in RFC 1378, this paper has shown a novel way to identify variables by employing graph theory and branching complexity analysis.  This approach worked with the sample application but needs additional work to be universally applicable to all semantic URL formats.  For instance, an application that has a small number of variables that equal the branching complexity of the URL structure could result in false positives with variable identification.

This research also showed that using abstract URLs to generate test cases was an effective and inexpensive way to discover all types of faults except data faults in the sample application and particularly well suited for template heavy client-side applications.  While it didn't find as many errors as the exhaustive or combinatorial approaches, with only 27 tests (0.0001\% of the exhaustive tests run) it found 19.5\%.  Using abstract URLs may be a good strategy when time is extremely limited.  A point of future work that may possibly yield better results with abstract URLs would be to increase the number of random variables used from a single variable to a statistically significant percentage of total data points available.

Also demonstrated in this paper was a convenient way to capture JavaScript exceptions by intercepting HTTP requests and responses via an embedded proxy server, and then injecting a JavaScript array in the \textless HEAD\textgreater\ section of each HTML page to capture any thrown exceptions.

Future work may extend testing to distributed machines and different client environments.  For example, the environment variables shown in Table~\ref{table:environmentVariables} could be combined with input variables to help catch compatibility faults, an ongoing concern with the various browser creators adoption rate of new HTML5 features.

\begin{table}[h]
	\centering
	\caption{Environment Variables.}
	\begin{tabular}{| l | l |}
		\hline
  OS 			& 	Browser		\\ \hline
  Linux 		& 	Chrome		\\ \hline
  OSX 		& 	Firefox		\\ \hline
  Windows 	&	IE			\\ \hline
			&	Safari		\\
		\hline
	\end{tabular}
\label{table:environmentVariables}
\end{table}

While I attempted to make the sample application as ``real world'' as possible, additional work is needed to make the testing tool better and more practical for rich web applications in real world scenarios.  An initial effort was made to implement ideas in Microsoft's PICT tool \cite{czerwonka2008pairwise}, such as making a better distinction between preparation and test generation, pre-combining related, hierarchical fields, and providing test generation guidelines by employing the new HTML5 data-* attributes.  As an example of future enhancements to the testing software, one could create a new ``data-exclude-test'' to keep a particular variable from being combined for test case generation.  Also, guidelines on associated form variables with attributes in the <fieldset> tags would allow for better control of test case generation with variables.  Integrating in with build tools would also be beneficial to keep those guideline attributes from being deployed to production.  More future work that would be beneficial would be better testing of all JavaScript event types and application state transitions that are not associated with an update to the URL fragment (i.e., an event triggered by the clicking of a non anchor tag that updates a value in an existing portion of the Document Object Model (DOM) structure).

Despite much future work left to be researched, this thesis demonstrates that test case generation using combinatorial coverage strategies in rich web applications, as in other application types, provides much benefit in identifying faults and should be explored further.