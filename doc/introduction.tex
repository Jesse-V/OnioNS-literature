
\chapter{\uppercase{Introduction}}

The Tor network is a second-generation onion routing system that aims to provide anonymity, privacy, and Internet censorship protection to its users. The Tor client software multiplexes all end-user TCP traffic through a series of relays on the Tor network, typically a carefully-constructed three-hop path known as a \textit{circuit}. Each relay in the circuit has its own encryption layer, so traffic is encrypted multiple times and then is decrypted in an onion-like fashion as it travels through the Tor circuit. As each relay sees no more than one hop in the circuit, in theory neither an eavesdropper nor a compromised relay can link the connection's source, destination, and content. Tor remains one of the most popular and secure tools to use against network surveillance, traffic analysis, and information censorship.

While the majority of Tor's usage is for traditional access to the Internet, Tor's routing scheme also supports anonymous websites, hidden inside Tor. Unlike the Clearnet, Tor does not contain a traditional DNS system for its websites; instead, hidden services are identified by their public key and can be accessed through Tor circuits. A client and the hidden service can thus communicate anonymously.
