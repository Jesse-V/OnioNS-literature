
\chapter{IMPLEMENTATION}

\section{Reference Implementation}

Alongside this publication, we provide a reference implementation of OnioNS. We utilize C++11, the Botan cryptographic library, the Standard Template Library's (STL) implementation of Mersenne Twister, and the libjsoncpp-dev library for JSON encoding. We develop in Linux Mint and compile for Ubuntu Vivid, Utopic, Trusty, and their derivatives. Our software is built on Canonical's Launchpad online build system and is available online at \url{https://github.com/Jesse-V/OnioNS}. 

\section{Prototype Design}

We have developed an OnioNS prototype that implements the Domain Query protocol. In this initial prototype, we use a static server, a single fixed Create Record, and a hidden service that we deployed at onions55e7yam27n.onion. Although our implementation is primarily a separate software package, we made necessary modifications to the Tor client software to intercept the .tor pseudo-TLD, pass the domain to the OnioNS client over inter-process communication (IPC), and receive and lookup the returned hidden service address. We diagram the relationships between our prototype's components in Figure \ref{fig:prototypeDiagram}.

Our prototype's works as follows:

\begin{enumerate}
	\item The user enters in ``example.tor'' into the Tor Browser.
	\item The Tor Browser sends ``example.tor'' to Tor's SOCKS port for resolution.
	\item The Tor client intercepts ``*.tor'', places the lookup in a wait state, and sends ``example.tor'' to the OnioNS client over a named pipe.
	\item The OnioNS client communicates with Tor's SOCKS port and negotiates a connection to the static OnioNS server.
	\item The OnioNS client performs a level 0 Domain Query to the server.
	\item The server responds with the Create Record containing ``example.tor''.
	\item The client writes ``onions55e7yam27n.onion'' to the Tor client over another named pipe.
	\item The Tor client resumes the lookup and overrides the original ``example.tor'' lookup with ``onions55e7yam27n.onion''.
	\item The Tor client contacts the OnioNS hidden service and passes the webpage to the Tor Browser.
	\item The Tor Browser displays the website contents and preserves the ``example.tor'' domain name.
\end{enumerate}

\begin{figure}[htbp]
	\centering
	\includegraphics[width=0.75\textwidth]{images/LucidCharts/OnioNS_Prototype.pdf}
	\caption{Our OnioNS prototype involves five components: the Tor Browser, a modified Tor client, a OnioNS client, a OnioNS server, and the destination hidden service. The Tor Browser passes an unknown .tor domain to the OnioNS through the Tor software (shown in red) which resolves the domain anonymously over a Tor circuit (orange) to remote resolver. Finally, the Tor software contacts the destination server via the normal hidden service protocol (green). The Tor Browser communicates to the Tor client over its SOCKS port, while the OnioNS client communicates over named pipes (red) and Tor's SOCKS port (orange).}
	\label{fig:prototypeDiagram}
\end{figure}

\subsection{Challenges}

We encountered two significant challenges while implementing the prototype. 

Our first modification to the Tor software used blocking I/O for communication with the OnioNS client. This caused Tor's event loop to pause while the OnioNS was resolving the domain name. When the OnioNS client attempted to use Tor to construct a circuit to the OnioNS server, Tor could not respond as it was waiting on I/O. This resulted in an unresolvable deadlock. After collaboration with several Tor developers we migrated our Tor modification to libevent, a software library that enables asynchronous event. Libevent is heavily used throughout Tor, Google Chrome, ntpd, and other software. Libevent enabled the OnioNS client to communicate over Tor, communicate with the remote OnioNS resolver, and return the hidden service address to Tor. Libevent then fired a callback method to contact the hidden service.

Our second challenge was telling Tor to place the resolution of the domain on hold. Previously, Tor would attempt to interpret the .tor domain and fail the lookup almost instantaneously. To resolve this, we placed the resolution in a waiting state. Then when OnioNS resolved the domain, our Libevent callback resumed the lookup, passed in the initial state, and allowed the lookup to continue as if a hidden service address had been requested in the first place. This allowed the Tor Browser to view the destination hidden service under a .tor domain name.

\subsection{Performance}

We created a rudimentary form of the Record Generation and Client Verification protocols and conducted performance measurements on two different machines. We tested these protocols on Machine A, which has an Intel Core2 Quad Q9000 @ 2.00 GHz; and machine B, which has an Intel i7-2600K @ 4.3 GHz. We chose machines with this hardware in order to more accurately determine the performance with out target audience; Machines A and B represent low-end and medium-end consumer-grade computers, respectively. To minimize latency on our end, both machines were hosted on 1 Gbits connections at Utah State University. As the Record Generation protocol requires a hidden service private key, we created a hidden service and hosted it on Machine B, using Shallot, a vanity key generator, to generate a recognizable hidden service address, \url{http://onions55e7yam27n.onion}. Then on Machine A, we measured the time required to connect to our hidden service over OnioNS and over the more direct hidden service protocol.

We selected the parameters of scrypt such that it consumed 128 MB of RAM during operation. This is an affordable amount of RAM by even low-end consumer-grade machines. This RAM consumption scales linearly with the number of scrypt instances executed in parallel. We used all eight cores on Machine B to generate a Record for our hidden service and observed approximately 1 GB of RAM consumption, matching our expectations. We set the difficulty of the Create Record such that the Record Generation protocol took only a few minutes on average to conduct, but in the future we may change it so that the protocol takes several hours instead.

\subsubsection{Processing Time}

We measured the CPU wall time required for different parts of client-side protocols. We measured how long it takes the client to build a Record from a JSON-formatted textual string, which involves parsing and assembly of the various fields; the time to check the proof-of-work \emph{PoW}, and the time to check the \emph{recordSig} digital signature.

\renewcommand{\arraystretch}{1}
\begin{center}
    \begin{tabular}{ | c | c | c | c |}
    \hline
    \textbf{Description} & \textbf{A Time (ms)} & \textbf{B Time (ms)} & \textbf{Samples} \\
    \hline
    Parsing JSON into a Record & 0.052 & 0.0238 & 100 \\ \hline
	Scrypt check & 896.369 & 589.926 & 25 \\ \hline
	Check of $ S_{d}(m, r) $ RSA signature & 0.06304 & 0.0267 & 200 \\
	\hline
    \end{tabular}
\end{center}

Machine B, as expected, performed much faster than Machine A at all of these tasks. Parsing and signature checks both took trivial time, though the total time was dominated by the single iteration of scrypt. Record Validation protocol is single threaded and consumes 128 MB of RAM due to scrypt.

\subsubsection{Latency}

We compared the load latency between an OnioNS domain name with a traditional hidden service address. Our tests measured the time between when a user entered in ``example.tor'' into the Tor Browser to the time when the browser first began to load our hidden service webpage. We also tested \url{http://onions55e7yam27n.onion}, the destination of ``example.tor''. We performed our experiment 15 times with a different client-side Tor circuit for each by restarting Tor at each iteration. To prevent browser-side caching, we restarted the Tor Browser between tests as well. 

\renewcommand{\arraystretch}{1}
\begin{center}
    \begin{tabular}{ | c | c | c | c |}
    \hline
    \textbf{Lookup} & \textbf{Fastest Time} & \textbf{Slowest Time} & \textbf{Mean Time} \\
    \hline
    .tor & 6.1 & 8.5 & 7.1 \\ \hline
	.onion & 9.3 & 12.2 & 10.2 \\
	\hline
    \end{tabular}
\end{center}

The latency is circuit-dependent and heavily depends on the speed of each Tor router and the distance between them. To avoid the latency cost whenever possible, we implemented a DNS cache into the OnioNS client-side software to allow subsequent queries to be resolved locally, avoiding unnecessary remote lookups.

\subsection{Future Work}

We have several plans for future work. First, we will port the OnioNS client to Windows and migrate the Tor-OnioNS IPC from Unix named pipes to TCP sockets. Although named pipes work reliably in Unix-like operating systems, they are not easily compatible in the Windows family of operating systems. Second, we will find more optimal parameters for scrypt to increase performance and work to reduce the latency. Finally, we will expand our implementation and collaborate with Tor developers to merge it into Tor's repositories when it is complete.


%\section{Experimentation}
%
%
%\emph{Todo: I will carry out experiments in test deployments of the Tor network and see what the demand is. I anticipate it being relatively lightweight. The proof-of-work will almost certainly be the computational bottleneck.}
%
%\emph{Bandwidth, CPU, RAM, latency for clients to be determined...}

%demand on participating nodes to be determined...

%Unlike Namecoin, OnionNS' \emph{page}-chain is of $ L $ days in maximal length. This serves two purposes:

%\begin{enumerate}
%	\item Causes domain names to expire, which reduced the threat of name squatting.
%	\item Prevents the data structure from growing to an unmanageable size.
%\end{enumerate}

%\subsection{Proof-of-work}
%
%\emph{What are the optimal parameters for scrypt? What are the implications of setting it too high or too low?}
%
%\emph{How much bandwidth and time does this take?}
%
%\emph{How does the bandwidth and CPU load scale in response to a larger Quorum? Assuming reasonable clockskew, how off will the exchanges be? Are there any race conditions that need to be resolved?}
%
%\emph{Queres: How long does this take, and how can we improve efficiency? Can clients on even low-end hardware calculate the proof-of-work?}
%
%
%\section{Results}
%
%
%\emph{This will be expanded/rewritten once I finish implementation and deploy it in test/prototyping networks. I plan on using Chutney.}
