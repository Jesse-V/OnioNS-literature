
\chapter{CONCLUSION}

The security analysis strongly suggests that larger Quorums, such as 255 or 511, is superior choice. Figure ? suggests that even larger Quorums would be ideal. Furthermore, Figure ? suggests that a slow rotation with a large Quorum is ideal. However, Figure ? suggests that a fast rotation reduces the impact that a malicious Quorum could have on the system. Our performance analysis shows that a small Quorum reduces the load on all parties. We discard Quorum sizes of 512 and above: we consider the load too significant, and discard 31 and smaller: Figures ? and ? suggest that these Quorums are too small, and Figures ? and ? show that 63 is similar to 31, so we discard that size as well. Between the choices of 127 and 255, we select 127 as the suggested Quorum size and reflect this in our reference implementation. Our stability analysis and Figure ? suggests 7 or 14 for rotation, we choose 7. 


We have introduced OnioNS, a Tor-powered distributed DNS that maps custom .tor domain names to traditional Tor .onion addresses. It enables hidden service operators to select a human-meaningful domain name and provide access to their service through that domain. We preserve the privacy and anonymity of both parties during registration, maintenance, and lookup, and furthermore allow Tor clients to verify the authenticity of domain names. Moreover, we rely heavily upon existing Tor infrastructure, which simplifies our design assumptions and narrows our threat model largely to attack vector already well-understood throughout the Tor literature.

The Pagechain is our novel distributed data structure which we introduce to resolve the difficulty of keeping large number of separate machines synchronized to a master database. While some other prominent distributed DNS schemes such as Namecoin use a blockchain, our Pagechain does utilize mining, but rather digital signatures which are already common throughout the Tor network. The Pagechain is also partially self-healing in the face of corruption or malicious modification, and each involved participant can verify the integrity of their local Pagechain themselves. OnioNS' Pagechain also has a fixed maximal length, which places an upper bound on the networking, computational, and storage requirements for all participants, a valuable efficiency gain especially noticeable long-term.

OnioNS achieves all three properties of Zooko's Triangle: it is distributed, allows hidden service operators to select meaningful domain names, and all parties can confirm for themselves the uniqueness of domain names in the database. We provide a reference implementation in C++ that should enable Tor developers to deploy OnioNS into the Tor network with minimal effort. We believe that OnioNS will be a useful abstraction layer that will significantly enhance the usability and the popularity of Tor hidden services. 